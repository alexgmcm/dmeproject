
\section*{todo}
\topic{Alpha}
\begin{itemize}
    \item add gui controls for request behavior, adding TimeSeries, \ldots
    \item fix: NAV-GRAPH mouse-mapping!!
    \item warum so mühsames chunk-loading? -> mobile/edge verbindung, sehr große zeitbereiche in kleiner ganularität = overkill, \ldots
    \item make DataProvider a real Library!
    \item \tip{ANFORDERUNGEN AN FEATURES ANPASSEN!!!}
    \item featuresOfInterest als mögliches weiteres Feature (nicht current core)
    \item \todo{NAV GRAPH fixen!!!!!}
    \item Quellen!  Quellen!  Quellen!  Quellen!  Quellen!  Quellen!  Quellen!
    \item granularity als object des Datenmodells?
    \item modularer CODE
    \item DEMO mode
    \item API in hinsicht auf performantes rendering (requestAnimationFrame, timeout, \ldots) entwerfen/entworfen
    \item DELETIONS nicht vorgesehen -> für anwendungsfall untypisch
    \item DEMO mode
    \item clean up folder structure; complete demo + minimaldemo
    \item silbentrennung (sloppy?)
\end{itemize}
\topic{Final}
\begin{itemize}
    \item Zusammenfassung / abstract
    \item Sections on ODD pages for printing!!
    \item \todo{\large Create High Quality Images!}
    \item Check usage of: \\emph \\textbf \\topic \\subtopic
    \item inlinecode vs emph
    \item Code anpassen
    \item make code listing floating figures
    \item update TESTS and DOCUMENTATION
    \item Formatierung von Quellenangaben
\end{itemize}
\topic{Begriffe}
\begin{itemize}
    \item Visualisierungstool vs Visualisierungsbibliothek
    \item Analyst vs Akteur vs User vs Benutzer
    \item RangeOfInterest vs. FieldOfView
    \item inlinecode erst nach definition, vorher emph?
    \item Datenpunkt vs Zeitreihendatenpunkt
    \item Datenmodell vs Model vs Domain Modell
    \item TimeRange -> Interval
    \item TimeRangeSet -> NormalizedIntervalSet
    \item gecached vs gecachet vs zwischengespeichert vs gebuffert vs im client vorhanden
    \item \textbf{Visuelle Exploration} vs Zeitreihenvisualisierung
    \item Zeitreihenmanagement vs Zeitreihen-Management
    \item Zeitreihe vs TimeSeries
    \item TimeRange vs Betrachtungszeitraum vs Zeitbereich vs Zeitfenster vs \ldots
    \item Mapping vs Abbildung/Abbilden
    \item Viewport-Zustand vs Viewportzustand vs ViewportState
\end{itemize}
