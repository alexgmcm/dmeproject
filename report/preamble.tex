\documentclass[11pt, a4paper]{article}
%\documentclass[10pt, a4paper]{report}
% Palatino for rm and math | Helvetica for ss | Courier for tt
% http://www.tug.dk/FontCatalogue/palatino/
\usepackage{mathpazo} % math & rm
\linespread{1.05}        % Palatino needs more leading (space between lines)
\usepackage[scaled]{helvet} % ss
\usepackage{courier} % tt

% SANS SERIF:
%\renewcommand*\familydefault{\sfdefault} %% Only if the base font of the document is to be sans serif

\usepackage[english]{babel}          % deutsch, deutsche Rechtschreibung
%\usepackage[german]{babel}          % deutsch, deutsche Rechtschreibung
\usepackage[T1]{fontenc}            % Umlaute und deutsches Trennen
\usepackage[utf8]{inputenc}         % Unicode Text

\usepackage[pdftex]{graphicx}       % to \includegraphics

\usepackage{mathtools}              % additional math commands
\usepackage{amssymb}                % math fonts
\usepackage{amsfonts}               % math fonts
                                    % math environment umlauts: \ddot{a}
                                    % (http://blog.mixable.de/latex-mathe-umgebung-mit-umlauten/)

\usepackage{tabularx}

\usepackage[hidelinks]{hyperref}    % \url [hide border around links]
\usepackage[table]{xcolor}               % \definecolor{name}{rgb}{0.5,0.5,0.5}

\usepackage{paralist}               % enumeration and itermization: \compactitem
\renewenvironment{itemize}{\begin{compactitem}}{\end{compactitem}}
\renewenvironment{enumerate}{\begin{compactenum}}{\end{compactenum}}

\usepackage{pdfpages}               % \includepdf[pages={1}]{myfile.pdf}

\usepackage{caption}
\usepackage{listings}           % code listings
\usepackage{verbatim}           % \begin{comment}

\usepackage{placeins}          % \FloatBarrier

\usepackage{alltt}              % HTML-<pre> equivalent

\usepackage{mdframed}           % framed figures

\usepackage{pgffor}             % \foreach \index in {1, ..., 5}{ \index }

\usepackage{wasysym}            % :) SMILIES!! \smiley \frownie

\usepackage[normalem]{ulem}     % underlining with line breaks \uline{..}

%\usepackage{ifthen}             % \ifthenelse{\equal{#1}{expected}}{IFTHEN:}{ELSE}

\usepackage[nottoc]{tocbibind} % show bibliography in table of contents 
% http://ftp.gwdg.de/pub/ctan/macros/latex/contrib/tocbibind/tocbibind.pdf

\usepackage{algorithm2e}        % AFTER package tocbibind!!

\usepackage{etoolbox}           % conditionals:
                                % \newtoggle{name} \toggletrue{name} \iftoggle{name}{if}{else}
\usepackage{ifthen}

\usepackage{mdframed}           % framed text \begin{mdframed}

%%%%%%%%%%%%%%%%%%%%%%%%%%%%%%%%%%%%%%%%%%%%%%%%%%%%%%%%%%%%%%%%%%%%%%%%%%%%%%%
% Colors
%%%%%%%%%%%%%%%%%%%%%%%%%%%%%%%%%%%%%%%%%%%%%%%%%%%%%%%%%%%%%%%%%%%%%%%%%%%%%%%
\definecolor{lightgray}{gray}{0.9}
%\definecolor{lightergray}{gray}{0.7}
\definecolor{darkgray}{rgb}{.4,.4,.4}
\definecolor{purple}{rgb}{0.65, 0.12, 0.82}

% Set custom colors
\definecolor{code}{gray}{0}
\definecolor{canvas}{gray}{0.96}
\definecolor{comment}{rgb}{0, 0.456, 0}
\definecolor{keyword}{rgb}{0.5, 0, 0.5}
\definecolor{lightergray}{rgb}{0.9, 0.9, 0.9}

%%%%%%%%%%%%%%%%%%%%%%%%%%%%%%%%%%%%%%%%%%%%%%%%%%%%%%%%%%%%%%%%%%%%%%%%%%%%%%%
% Page Layout / Formatting
% http://en.wikibooks.org/wiki/LaTeX/Page_Layout
%%%%%%%%%%%%%%%%%%%%%%%%%%%%%%%%%%%%%%%%%%%%%%%%%%%%%%%%%%%%%%%%%%%%%%%%%%%%%%%
\usepackage[a4paper,top=3cm,bottom=3cm,left=3.5cm,right=2.5cm]{geometry}

\usepackage{setspace}        % zeilenabstand
\onehalfspacing

\setlength{\parskip}{0.5em}  % abstand nach Absatz
\setlength{\parindent}{0em}  % keine einrückung nach Absatz

\sloppy                      % großzügige Formatierung, weniger Sibentrennungen


% change section numbering
%\renewcommand\thesubsection{(\alph{subsection})}
%\renewcommand\thesubsubsection{\arabic{subsubsection}.}

\hyphenation{
    no-ta-tion
}



% start \sections on odd pages [not working?]
%\let\oldsection\section
%\def\section{\cleardoublepage\oldsection}


%%%%%%%%%%%%%%%%%%%%%%%%%%%%%%%%%%%%%%%%%%%%%%%%%%%%%%%%%%%%%%%%%%%%%%%%%%%%%%%
% Header
%%%%%%%%%%%%%%%%%%%%%%%%%%%%%%%%%%%%%%%%%%%%%%%%%%%%%%%%%%%%%%%%%%%%%%%%%%%%%%%
\usepackage{fancyhdr}

\newcommand{\enableHeaders}{%
    \setlength{\headheight}{15.2pt}
    \pagestyle{fancy}

    \fancyhf{}
    \lhead{\bfseries\nouppercase\leftmark}
    \rhead{\nouppercase\rightmark}
    \cfoot{\bfseries\thepage}
}

\newcommand{\disableHeaders}{%
    \renewcommand{\headrulewidth}{0pt}
    \fancyhf{}
    \cfoot{\bfseries\thepage}
}

%%%%%%%%%%%%%%%%%%%%%%%%%%%%%%%%%%%%%%%%%%%%%%%%%%%%%%%%%%%%%%%%%%%%%%%%%%%%%%%
% Custom Commands
%%%%%%%%%%%%%%%%%%%%%%%%%%%%%%%%%%%%%%%%%%%%%%%%%%%%%%%%%%%%%%%%%%%%%%%%%%%%%%%
% make \title available via \thetitle
%\def\title#1{\gdef\@title{#1}\gdef\thetitle{#1}}

\newcommand{\HRule}{\rule{\linewidth}{0.5mm}}


\newcommand{\umlClass}{%
    \newpage
    \begin{figure}[ht]
        \begin{mdframed}
            \begin{tabbing}
                XX\=XX\=XXXXXXXXXXXXXX\kill
                \textbf{TimeSeries:} \\
                \> \emph{data structure:}\\
                \> \> string: \textbf{name}\\
                \> \emph{operations:}\\
                \> \> void \textbf{addRow}()

            \end{tabbing}
        \end{mdframed}
        \caption{\label{myfig} My caption.}
    \end{figure}
}

% Annotations
%%%%%%%%%%%%%%%%%%%%%%%%%%%%%%%%%%%%%
\newtoggle{annotations}
\toggletrue{annotations}
%\togglefalse{annotations}

\mdfdefinestyle{annotation}{%
    linecolor=red
}
\newcommand{\annotation}[1]{%
    \iftoggle{annotations}{%
        %\arr
        \begin{mdframed}[style=annotation]
            #1
        \end{mdframed}
    }{%
        %[HIDDEN]
    }

}
\newcommand{\emptyline}{
    \mbox{}\\
}


% Use Cases
%%%%%%%%%%%%%%%%%%%%%%%%%%%%%%%%%%%%%

% Counter
\newcounter{useCases}
%\setcounter{requirements}{1}
\newcounter{subUseCases}

\newcommand{\ucLabel}[1]{%
    {[UC.#1]}%
    %[REQ\_#1]
}
\newcommand{\useCase}[2]{%
    \refstepcounter{useCases}
    \label{uc:#2}
    %\textbf{\sffamily #1 \ucLabel{\arabic{useCases}}}%
    \textbf{\sffamily \ucLabel{\arabic{useCases}} #1}%

}
\newcommand{\subUseCase}[2]{%
    \refstepcounter{subUseCases}
    \label{uc:#2}
    \textbf{\sffamily \ucLabel{\arabic{useCases}.\arabic{subUseCases}} #1}%
}
\newcommand{\ucref}[1]{%
    \ucLabel{\ref{uc:#1}}
}

% Table
%%%%%%%%%%%%%%%%%%%%%%%%%%%%%%%%%%%%%
\newcommand{\alternatingTable}[4]{%
%\begin{center}
%\end{center}
    \emptyline
    {\sffamily \useCase{#1}{#2}}

    #3
    %\begin{table}[h!t]
        %\caption{#1}
        \begin{center}
            \sffamily
            \rowcolors{1}{}{lightgray}
            \begin{tabular}{l  | p{10cm}}
                \hline
                #4
                \hline
            \end{tabular}
        \end{center}
    %\end{table}
}
\newcommand{\subTableUseCase}[4]{%
%\begin{center}
%\end{center}
    \emptyline
    {\sffamily \subUseCase{#1}{#2}}

    #3
    %\begin{table}[h!t]
        %\caption{#1}
        \begin{center}
            \sffamily
            \rowcolors{1}{}{lightgray}
            \begin{tabular}{l  | p{10cm}}
                \hline
                #4
                \hline
            \end{tabular}
        \end{center}
    %\end{table}
}

% Requirements
%%%%%%%%%%%%%%%%%%%%%%%%%%%%%%%%%%%%%
\newcounter{requirements}
%\setcounter{requirements}{1}

\newcommand{\reqlabel}[1]{%
    [REQ#1]%
    %[REQ\_#1]
}
\newcommand{\requirement}[2]{%
    \refstepcounter{requirements}
    \label{req:#2}
    %\paragraph{\large #1 \reqlabel{\arabic{requirements}}}\mbox{}\\
    \textbf{#1 \reqlabel{\arabic{requirements}}}%
}
\newcommand{\reqref}[1]{%
    \reqlabel{\ref{req:#1}}
}

\newcommand{\req}[1]{%
    \textbf{\color{blue!60}#1}%
}
%%%%%%%%%%%%%%%%%%%%%%%%%%%%%%%%%%%%%

\newcommand{\topic}[1]{%
    %\paragraph{\large #1}\mbox{}\\
    \paragraph{\textbf{#1}}\mbox{}\\
    %\paragraph{$\rightarrow$\large #1}\mbox{}\\
%    \begin{tabular}{@{\hspace{3ex}}p{42em}}
%        \paragraph{\large #1}\mbox{}\\
%    \end{tabular}
}
\newcommand{\subtopic}[1]{%
    \paragraph{\textbf{#1}}\mbox{}\\
}

\newcommand{\todo}[1]{%
    \emph{\color{red!50}#1}%
}

\newcommand{\tocite}[1]{%
    \emph{\color{blue!35}[REF: #1]}%
}

\newcommand{\omg}[1]{%
    \mbox{
    \hspace{0.4\textwidth} \color{red}#1%
    }
}

\newcommand{\tip}[1]{%
    \emph{\color{blue!55}#1}
}

% inline code
\newcommand{\ic}[1]{%
    {\ttfamily #1}%
%    {%
%        \setlength{\fboxsep}{1pt}%
%        \fcolorbox{lightgray}{lightergray}{%
%            \lstinline[basicstyle=\ttfamily]{#1}%
%        }%
%    }%
}

\newcommand{\arr}{%
    $\rightarrow$
}
\newcommand{\darr}{%
    \\$\rightarrow\rightarrow$
}
\newcommand{\codedef}[3]{%
    \ic{#1} \emph{(#2)}\\
    #3
}

\newcommand{\nutshell}[1]{%

    \HRule

    \emph{#1}

    \HRule

}

\newcommand{\img}[4][1]{%
    %\abb[scale]{path}{caption}{label}
    \begin{figure}[h!]
        \centering
        \includegraphics[width=#1\textwidth]{#2}
        %\includegraphics[scale=#1]{#2}
        \caption{#3}
        \label{img:#4}
    \end{figure}
}
\newcommand{\imgref}[1]{%
    \emph{\textbf{Abb. \ref{img:#1}}}%
    %\emph{\textbf{Abbildung \ref{img:#1}}}%
    %Abbildung \ref{img:#1}%
    %\mbox{Abb. \ref{img:#1}}
}
\newcommand{\lstref}[1]{%
    \emph{\textbf{Listing \ref{code:#1}}}%
}

%%%%%%%%%%%%%%%%%%%%%%%%%%%%%%%%%%%%%%%%%%%%%%%%%%%%%%%%%%%%%%%%%%%%%%%%%%%%%%%
% custom environments
%%%%%%%%%%%%%%%%%%%%%%%%%%%%%%%%%%%%%%%%%%%%%%%%%%%%%%%%%%%%%%%%%%%%%%%%%%%%%%%
\newtheorem{define}{Definition}         % use as environment: \begin{define}

\newenvironment{reviewx}
{% This is the begin code 
    \color{orange!70} % 50percent red
    %\begin{it}     % italic
}
{% This is the end code
    %\end{it}
}



%%%%%%%%%%%%%%%%%%%%%%%%%%%%%%%%%%%%%%%%%%%%%%%%%%%%%%%%%%%%%%%%%%%%%%%%%%%%%%%
% Configure Listings
%%%%%%%%%%%%%%%%%%%%%%%%%%%%%%%%%%%%%%%%%%%%%%%%%%%%%%%%%%%%%%%%%%%%%%%%%%%%%%%


\lstdefinelanguage{javascript}{
    keywords={typeof, new, true, false, catch, function, return, null, catch, switch, var, if, for, in, while, do, else, case, break},
    ndkeywords={class, export, boolean, throw, implements, import, this},
    sensitive=false,
    comment=[l]{//},
    morecomment=[s][\color{blue}\ttfamily]{/*}{*/},
    morestring=[b]',
    morestring=[b]"
}


\lstdefinestyle{jsstyle}{
    belowcaptionskip=1\baselineskip,
    breaklines=true,
    captionpos=b,
    frame=L,
    xleftmargin=\parindent,
    language=C,
    showstringspaces=false,
    basicstyle=\footnotesize\ttfamily,
    keywordstyle=\bfseries\color{green!40!black},
    commentstyle=\itshape\color{purple!40!black},
    identifierstyle=\color{blue},
    stringstyle=\color{orange},
}

% Parameters for formatting code (optional, affects all code listings)
% http://en.wikibooks.org/wiki/LaTeX/Source_Code_Listings
\lstset{
    language=javascript,
    captionpos=b,
    backgroundcolor=\color{canvas},
    basicstyle=\ttfamily\small\color{code},
    commentstyle=\color{comment},
    identifierstyle=\color{black},
    keywordstyle=\color{keyword}\bfseries,
    ndkeywordstyle=\color{keyword}\bfseries,
    stringstyle=\color{blue}\ttfamily,
    frame=single,
    numbers=left,
    caption={\lstname}
}

\newcommand{\code}[2]{
    %\begin{minipage}{\textwidth}
        \lstinputlisting[caption=#1, escapechar=, style=jsstyle]{#2}
    %\end{minipage}
}


%%%%%%%%%%%%%%%%%%%%%%%%%%%%%%%%%%%%%%%%%%%%%%%%%%%%%%%%%%%%%%%%%%%%%%%%%%%%%%
% FINALIZE
%
% -> hide annotations
% -> disable \todo \tip style
%%%%%%%%%%%%%%%%%%%%%%%%%%%%%%%%%%%%%%%%%%%%%%%%%%%%%%%%%%%%%%%%%%%%%%%%%%%%%%
\newcommand{\finalize}{%
    \renewcommand{\todo}[1]{##1}
    \renewcommand{\tip}[1]{##1}
    \renewcommand{\tocite}[1]{}
    \renewcommand{\req}[1]{##1}
    \togglefalse{annotations}
}

%\finalize
